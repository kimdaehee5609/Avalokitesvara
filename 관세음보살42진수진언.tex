%	-------------------------------------------------------------------------------
% 
%
%
%
%
%
%
%
%
%
%	-------------------------------------------------------------------------------

	\documentclass[12pt, a4paper, oneside]{book}
%	\documentclass[12pt, a4paper, landscape, oneside]{book}

		% --------------------------------- 페이지 스타일 지정
		\usepackage{geometry}
%		\geometry{landscape=true	}
		\geometry{top 		=10em}
		\geometry{bottom	=10em}
		\geometry{left		=8em}
		\geometry{right		=8em}
		\geometry{headheight	=4em} % 머리말 설치 높이
		\geometry{headsep		=2em} % 머리말의 본문과의 띠우기 크기
		\geometry{footskip		=4em} % 꼬리말의 본문과의 띠우기 크기
% 		\geometry{showframe}
	
%		paperwidth 	= left + width + right (1)
%		paperheight 	= top + height + bottom (2)
%		width 		= textwidth (+ marginparsep + marginparwidth) (3)
%		height 		= textheight (+ headheight + headsep + footskip) (4)



		%	===================================================================
		%	package
		%	===================================================================
%			\usepackage[hangul]{kotex}				% 한글 사용
			\usepackage{kotex}						% 한글 사용
			\usepackage[unicode]{hyperref}			% 한글 하이퍼링크 사용
			\usepackage{amssymb,amsfonts,amsmath}	% 수학 수식 사용

			\usepackage{scrextend}					% 
		
			\usepackage{enumerate}			%
			\usepackage{enumitem}			%
			\usepackage{tablists}			%	수학문제의 보기 등을 표현하는데 사용
										%	tabenum


		% ------------------------------ table 
			\usepackage{longtable}			%
			\usepackage{tabularx}			%
			\usepackage{tabu}				%



		% ------------------------------ 
			\usepackage{setspace}			%
			\usepackage{booktabs}			% table
			\usepackage{color}				%
			\usepackage{multirow}			%
			\usepackage{boxedminipage}		% 미니 페이지
			\usepackage[pdftex]{graphicx}	% 그림 사용
			\usepackage[final]{pdfpages}	% pdf 사용
			\usepackage{framed}			% pdf 사용
			
			\usepackage{fix-cm}	
			\usepackage[english]{babel}
	
			\usepackage{tikz}%
			\usetikzlibrary{arrows,positioning,shapes}
			%\usetikzlibrary{positioning}
			


		% Package --------------------------------- 

			\usepackage{tablists}				%



		% Package --------------------------------- 

			\usepackage{exsheets}				%

			\SetupExSheets{solution/print=true}
			\SetupExSheets{question/type=exam}
			\SetupExSheets[points]{name=point,name-plural=points}


		% --------------------------------- 페이지 스타일 지정

		\usepackage[Sonny]		{fncychap}

			\makeatletter
			\ChNameVar	{\Large\bf}
			\ChNumVar	{\Huge\bf}
			\ChTitleVar		{\Large\bf}
			\ChRuleWidth	{0.5pt}
			\makeatother

%		\usepackage[Lenny]		{fncychap}
%		\usepackage[Glenn]		{fncychap}
%		\usepackage[Conny]		{fncychap}
%		\usepackage[Rejne]		{fncychap}
%		\usepackage[Bjarne]		{fncychap}
%		\usepackage[Bjornstrup]{fncychap}

		\usepackage{fancyhdr}
		\pagestyle{fancy}
		\fancyhead{} % clear all fields
		\fancyhead[LO]{\footnotesize \leftmark}
		\fancyhead[RE]{\footnotesize \leftmark}
		\fancyfoot{} % clear all fields
		\fancyfoot[LE,RO]{\large \thepage}
		%\fancyfoot[CO,CE]{\empty}
		\renewcommand{\headrulewidth}{1.0pt}
		\renewcommand{\footrulewidth}{0.4pt}
	
	
	

	
		%	=======================================================================================
		% 	tritlesec package
		% 	
		% 	
		% 	------------------------------------------------------------------ section 스타일 지정
	
			\usepackage{titlesec}
		
		% 	----------------------------------------------------------------- section 글자 모양 설정
			\titleformat*{\section}					{\large\bfseries}
			\titleformat*{\subsection}				{\normalsize\bfseries}
			\titleformat*{\subsubsection}			{\normalsize\bfseries}
			\titleformat*{\paragraph}				{\normalsize\bfseries}
			\titleformat*{\subparagraph}				{\normalsize\bfseries}
	
		% 	----------------------------------------------------------------- section 번호 설정
%			\renewcommand{\thepart}				{\arabic{part}.}
			\renewcommand{\thepart}				{}
%			\renewcommand{\thechapter}			{\arabic{chapter}.}
			\renewcommand{\thechapter}			{}
			\renewcommand{\thesection}			{\arabic{section}.}
			\renewcommand{\thesubsection}			{\thesection\arabic{subsection}.}
			\renewcommand{\thesubsubsection}		{\thesubsection\arabic{subsubsection}}

		% 	----------------------------------------------------------------- section 페이지 나누기 설정
			\let\stdsection\section
			\renewcommand\section{\newpage\stdsection}



		%	--------------------------------------------------------------------------------------- 
		% 	\titlespacing*{commandi} {left} {before-sep} {after-sep} [right-sep]		
		% 	left
		%	before-sep		:  수직 전 간격
		% 	after-sep	 	:  수직으로 후 간격
		%	right-sep

			\titlespacing*{\section} 			{0pt}{1.0em}{1.0em}
			\titlespacing*{\subsection}	  		{0ex}{1.0em}{1.0em}
			\titlespacing*{\subsubsection}		{0ex}{1.0em}{1.0em}
			\titlespacing*{\paragraph}			{0em}{1.5em}{1.0em}
			\titlespacing*{\subparagraph}		{4em}{1.0em}{1.0em}
	
		%	\titlespacing*{\section} 			{0pt}{0.0\baselineskip}{0.0\baselineskip}
		%	\titlespacing*{\subsection}	  		{0ex}{0.0\baselineskip}{0.0\baselineskip}
		%	\titlespacing*{\subsubsection}		{6ex}{0.0\baselineskip}{0.0\baselineskip}
		%	\titlespacing*{\paragraph}			{6pt}{0.0\baselineskip}{0.0\baselineskip}
	

		% --------------------------------- recommend		섹션별 페이지 상단 여백
		\newcommand{\SectionMargin}			{\newpage  \null \vskip 2cm}
		\newcommand{\SubSectionMargin}		{\newpage  \null \vskip 2cm}
		\newcommand{\SubSubSectionMargin}		{\newpage  \null \vskip 2cm}


	

		%	--------------------------------------------------------------------------------------- 
		% 	toc 설정  - table of contents
		% 	
		% 	
		% 	----------------------------------------------------------------  문서 기본 사항 설정
			\setcounter{secnumdepth}{3} 		% 문단 번호 깊이
			\setcounter{tocdepth}{1} 			% 문단 번호 깊이 - 목차 출력시 출력 범위

			\setlength{\parindent}{0cm} 		% 문서 들여 쓰기를 하지 않는다.


		%	--------------------------------------------------------------------------------------- 
		% 	mini toc 설정
		% 	
		% 	
		% 	--------------------------------------------------------- 장의 목차  minitoc package
			\usepackage{minitoc}

			\setcounter{minitocdepth}{1}    	%  Show until subsubsections in minitoc
%			\setlength{\mtcindent}{12pt} 	% default 24pt
			\setlength{\mtcindent}{24pt} 	% default 24pt

		% 	--------------------------------------------------------- part toc
		%	\setcounter{parttocdepth}{2} 	%  default
			\setcounter{parttocdepth}{0}
		%	\setlength{\ptcindent}{0em}		%  default  목차 내용 들여 쓰기
			\setlength{\ptcindent}{0em}         


		% 	--------------------------------------------------------- section toc

			\renewcommand{\ptcfont}{\normalsize\rm} 		%  default
			\renewcommand{\ptcCfont}{\normalsize\bf} 	%  default
			\renewcommand{\ptcSfont}{\normalsize\rm} 	%  default


		%	=======================================================================================
		% 	tocloft package
		% 	
		% 	------------------------------------------ 목차의 목차 번호와 목차 사이의 간격 조정
			\usepackage{tocloft}

		% 	------------------------------------------ 목차의 내어쓰기 즉 왼쪽 마진 설정
			\setlength{\cftsecindent}{2em}			%  section

		% 	------------------------------------------ 목차의 목차 번호와 목차 사이의 간격 조정
			\setlength{\cftsecnumwidth}{2em}		%  section
		
% 	============================================================================== itemi Global setting


	%	-------------------------------------------------------------------------------
	%		Vertical spacing
	%	-------------------------------------------------------------------------------
		\setlist[itemize]{topsep=0.0em}			% 상단의 여유치
		\setlist[itemize]{partopsep=0.0em}			% 
		\setlist[itemize]{parsep=0.0em}			% 
%		\setlist[itemize]{itemsep=0.0em}			% 
		\setlist[itemize]{noitemsep}				% 
		
	%	-------------------------------------------------------------------------------
	%		Horizontal spacing
	%	-------------------------------------------------------------------------------
		\setlist[itemize]{labelwidth=1em}			%  라벨의 표시 폭
		\setlist[itemize]{leftmargin=8em}			%  본문 까지의 왼쪽 여백  - 4em
		\setlist[itemize]{labelsep=3em} 			%  본문에서 라벨까지의 거리 -  3em
		\setlist[itemize]{rightmargin=0em}			% 오른쪽 여백  - 4em
		\setlist[itemize]{itemindent=0em} 			% 점 내민 거리 label sep 과 같은면 점위치 까지 내민다
		\setlist[itemize]{listparindent=3em}		% 본문 드려쓰기 간격


		\setlist[itemize]{ topsep=0.0em, 			%  상단의 여유치
					partopsep=0.0em, 		%  
					parsep=0.0em, 
					itemsep=0.0em, 
					labelwidth=1em, 
					leftmargin=2.5em,
					labelsep=2em,			%  본문에서 라벨 까지의 거리
					rightmargin=0em,		% 오른쪽 여백  - 4em
					itemindent=0em, 		% 점 내민 거리 label sep 과 같은면 점위치 까지 내민다
					listparindent=0em}		% 본문 드려쓰기 간격

%		\begin{itemize}

	%	-------------------------------------------------------------------------------
	%		Label
	%	-------------------------------------------------------------------------------
		\renewcommand{\labelitemi}{$\bullet$}
		\renewcommand{\labelitemii}{$\bullet$}
%		\renewcommand{\labelitemii}{$\cdot$}
		\renewcommand{\labelitemiii}{$\diamond$}
		\renewcommand{\labelitemiv}{$\ast$}		

%		 \renewcommand{\labelitemi}{$\blacksquare$}   	% 사각형 - 찬것
%		 \renewcommand\labelitemii{$\square$}		% 사각형 - 빈것	
			


		
		% --------------------------------- 	줄간격 설정
		\doublespace
%		\onehalfspace
%		\singlespace
		
		

			
			
		\usepackage[parfill]{parskip}
		\setlength{\parskip}{1em}
		
% ------------------------------------------------------------------------------
% Begin document (Content goes below)
% ------------------------------------------------------------------------------
	\begin{document}
	
			\dominitoc
			

			\title{ 관세음보살 42진수 진언 }
			\author{ 김대희}
			\date{2020년  5월 15일}
			\maketitle


			\tableofcontents
%			\listoffigures
%			\listoftables



%			\begin{itemize}
%			\item []
%			\end{itemize}



% ================================================= chapter 	====================
	\chapter{관세음보살 42수주의 유래}


	% -------------------------------------- page -------------------
	%	\nomtcrule         		% removes rules = horizontal lines
	%	\nomtcpagenumbers  % remove page numbers from minitocs
		\newpage
		\minitoc				% Creating an actual minitoc
	%	\doublespace




\section{ 관세음보살 42수주의 유래 }


 『불설천수천안관세음보살광대원만무애대비심다라니경』에서는 관세음보살의 40수만 열거되고 있으며,  실제로 그 각각의 손에 따른 진언은 나오지 않는다.
 그러나 『천수천안관세음보살대비심다라니』에는 41가지의 손 도상과 함께 그에 따라 다양한 진언이 갖추어져 있다.
 하지만 『천수천안관세음보살대비심다라니』​의 열거 순서와 일치하는 것도 아니다.
 후대에 이르러서 이러한 것들을 종합하여 관세음보살의 42수주로 정리되었다.


 42수주는 천수대비주와는 내용이 다르며, 구체적인 경우에 따른 세부적인 42가지 진언을 설하고 있다.
 예컨대 이런 경우에는 이런 진언을 외우고, 저런 경우에는 저런 진언을 외우라는 식으로 매우 구체적인 상황을 열거하고 있는데, 그것이 모두 42가지이다.

 천수대비주[신묘장구대다라니]는 총주이기 때문에 42수주의 공덕을 모두 한 데 갈무리 하고 있다.
 하지만 관세음보살 42수주는 중생 저마다 소원과 인연, 그리고 근기 등에 따라서 간단히 염송할 수 있도록 구체적인 필요를 구분하여 세세히 충족시킬 수 있도록 제시한 것이다.
 따라서 항상 42가지 주문을 모두 외우는 것이 아니다. 
 42수주 가운데 자신이 꼭 필요한 경우의 진언을 골라서 그 목적과 때에 따라 온 마음을 다해서 염송하는 것이 기본이다.


 42수주는 예로부터 매우 간명하지만 미묘한 효험이 있는 진언이므로 누구든지 쉽게 혼자서도 실천할 수 있을 것이다.
 예컨대 입시를 앞둔 수험생이라면 "옴 아하라 살바미냐 다라 바니제 사바하"라는 37수 관세음보살 보경수 진언을 시시때때로 염송하는 것이 좋을 것이다.
 심리적 안정은 물론이고, 정신 집중에도 탁월한 효험을 볼 것이다.

 이처럼 관세음보살의 42수 진언은 누구든지 언제 어디서나 가장 간단히 할 수 있는 진언 수행이기 때문에 보편적인 기도 수행으로 잘 알려져 있다.
 각자 자신의 상황과 소원에 맞는 수주를 때에 따라 골라서 염송한다면 가장 신속하게 관세음보살의 가피 영험을 체험할 수 있는 지름길 수행법이 될 것이다.
 이러한 점에서 볼 때, 일상생활이 바쁘면 바쁠수록 성취하고 싶은 욕망도 커져만 가는 현대인들에게 관세음보살의 42수 진언 수행은 매우 적절한 수행법이 될 것이다.

                 -수락산 도안사[혜자 스님] 관세음보살 사십이수주 독송 기도집 中-





출처: https://buddha-mind.tistory.com/120 [부처님 마음 卍 불교의 향기]



% ------------------------------------------------------------------------------ section
	\section{ 불설 천수천안 관세음보살 광대원만무애대비심 다라니경 }



% ------------------------------------------------------------------------------ section
	\section{ 천수천안 관세음보살 대비심 다라니 }



% ================================================= chapter 	====================
	\chapter{관세음보살 42수주 해설 }


	% -------------------------------------- page -------------------
	%	\nomtcrule         		% removes rules = horizontal lines
	%	\nomtcpagenumbers  % remove page numbers from minitocs
		\newpage
		\minitoc				% Creating an actual minitoc
	%	\doublespace



% ------------------------------------------------------------------------------ section
	\section{ 관세음보살 42수주 해설}



% ------------------------------------------------------------------------------ section
	\section{ 옴 }


% ------------------------------------------------------------------------------ section
	\section{ 훔 }



% ------------------------------------------------------------------------------ section
	\section{ 바탁 }



% ------------------------------------------------------------------------------ section
	\section{ 사바하 }






% ================================================= chapter 	====================
	\chapter{관세음보살 42수주}


	% -------------------------------------- page -------------------
	%	\nomtcrule         		% removes rules = horizontal lines
	%	\nomtcpagenumbers  % remove page numbers from minitocs
		\newpage
		\minitoc				% Creating an actual minitoc
	%	\doublespace




\section{1수. 관세음보살 여의주수 진언}

			\begin{itemize}
			\item 물질적 풍요로움과 안락한 생활을 원할 때
			\item 옴 바아라 바다라 훔 바탁
			\item 범어 - 옴 봐즈라 봐따라 훔파트
			\item Om vajra vatara humphat
			\end{itemize}





\section{2수. 관세음보살 견색수 진언 }

			\begin{itemize}
			\item 온갖 불안 속에서 마음이 편해지기를 원할 때
			\item 옴 기니라나 모나라 훔 바탁
			\item 범어 - 옴 끼를라라 모드라 훔파트 
			\item Om kirlara modra humphat
			\end{itemize}





\section{3수. 관세음보살 보발수 진언 }

			\begin{itemize}
			\item 온갖 아픈 병이 낫기를 원할 때
			\item 옴 기리기리 바라아 훔 바탁
			\item 범어 - 옴 끼르끼르 바즈라 훔파트
			\item Om kirkir vajra humhpat
			\end{itemize}




\section{4수. 관세음보살 보검수 진언}

			\begin{itemize}
			\item 모든 잡귀들을 항복시키기를 원할 때
			\item 옴 제세제야 도미니 도제 삿다야 훔 바탁
			\item 범어 - 옴 떼세떼자 뚜뷔니 뚜데 사따야 훔파트 
			\item Om teseteja tuvini tude satdhaya humphat
			\end{itemize}





\section{5수. 관세음보살 발절라수 진언}

			\begin{itemize}
			\item 모든 잡귀들을 항복 시키기를 원할 때
			\item 옴 이베이베 이파야 마하 시리예 사바하
			\item 범어 - 옴 디베디베 디뺘 마하 스례(스르예) 스바하
			\item Om dibhedibhe dipya maha srye svahat
			\end{itemize}



 


\section{6수. 관세음보살 금강저수 진언}

			\begin{itemize}
			\item 모든 적을 항복시키기를 원할 때
			\item 옴 바아라 아니바라 닙다야 사바하
			\item 범어 - 옴 바즈라 그니 쁘라 딥따야 스바하
			\item Om vajra gni pra diptaya svaha
			\end{itemize}





\section{7수. 관세음보살 시무외수 진언}

			\begin{itemize}
			\item 모든 적을 항복시키기를 원할 때
			\item 옴 아라나야 훔 바탁
			\item 범어 - 옴 즈라나야 훔파트
			\item Om jranaya humphat
			\end{itemize}





\section{8수. 관세음보살 일정마니수 진언}

			\begin{itemize}
			\item 눈이 어두워져 밝은 눈을 갖기를 원할 때
			\item 옴 도비가야 도비바라 바리니 사바하
			\item 범어 - 옴 뚜삐까야 뚜삐뿌라 바르디 스바하
			\item Om tupikaya tupipra vardi svaha
			\end{itemize}





\section{9수. 관세음보살 월정마니수 진언}

			\begin{itemize}
			\item 심한 열병을 앓아서 낫기를 원할 때
			\item 옴 소싯지 아리 사바하
			\item 범어 - 옴 슈시디 그르 스바하
			\item Om susidhi gr svaha
			\end{itemize}





\section{10수. 관세음보살 보궁수 진언}

			\begin{itemize}
			\item 승진하거나 높은 관직을 얻기를 원할 때
			\item 옴 아지미례 사바하
			\item 범어 - 옴 아차(짜)비레 스바하
			\item Om acavire svaha 
			\end{itemize}



 


\section{11수. 관세음보살 보전수 진언 }

			\begin{itemize}
			\item 빨리 착하고 좋은 친구들을 많이 만나기를 원할 때
			\item 옴 가마라 사바하
			\item 범어 - 옴 까마라 스바하
			\item Om kamala svaha
			\end{itemize}





\section{12수. 관세음보살 양류지수 진언}

			\begin{itemize}
			\item 몸에 생긴 갖가지 병이 모두 낫기를 원할 때
			\item 옴 소싯지 가리바리 다남타 목다에 바아라 바아라 반다 하나하나 훔 바탁
			\item 범어 - 옴 슈싯디 까르바르타남타 묵따예 바즈라 바즈라 반다 하나하나 훔파트
			\item Om susitdhi karvartanamta muktaye vajra vajra vandha hanahana humphat
			\end{itemize}





\section{13수. 관세음보살 백불수 진언}

			\begin{itemize}
			\item 모든 나쁜 장애와 곤란을 없애기를 원할 때
			\item 옴 바나미니 바나바제 모하야 아아 모하니 사바하
			\item 범어 - 옴 빠드미니 바가바떼 모하야 자가 모하니 스바하
			\item Om padmini bhagavate mohaya jaga mohani svaha
			\end{itemize}





\section{14수. 관세음보살 보병수 진언}

			\begin{itemize}
			\item 모든 가족과 친족들이 원만하게 화합하기를 원할 때
			\item 옴 아례 삼만염 사바하
			\item 범어 - 옴 그레 삼맘얌 스바하(스와하)
			\item Om gre sammamyam svaha
			\end{itemize}





\section{15수. 관세음보살 방패수 진언}

			\begin{itemize}
			\item 어떤 동물이나 맹수로부터 피해를 당하지 않기를 원할 때
			\item 옴 약삼나나야 전나라 다노발야 바사바사 사바하
			\item 범어 - 옴 야크삼 나다야 쉬찬드라 다두빠르야빠샤 빠샤 스바하
			\item Om yaksam nadaya scandra dhaduparyapasa pasa svaha
			\end{itemize}



 


\section{16수. 관세음보살 월부수 진언}

			\begin{itemize}
			\item 언제 어디서나 관재를 당하지 않기를 원할 때
			\item 옴 미라야 미라야 사바하
			\item 범어 - 옴 미라야 미라야 스바하
			\item Om miraya miraya svaha
			\end{itemize}





\section{17수. 관세음보살 옥환수 진언}

			\begin{itemize}
			\item 남녀불문하고 좋은 친구나 동료를 갖고자 원할 때
			\item 옴 바나맘 미라야 사바하
			\item 범어 - 옴 빠드맘 미라야 스바하(스와하)
			\item Om padmam miraya svaha
			\end{itemize}





\section{18수. 관세음보살 백련화수 진언}

			\begin{itemize}
			\item 수많은 공을 세우고 온갖 공덕을 이루기를 원할 때
			\item 옴 바아라 미라야 사바하
			\item 범어 - 옴 바(와)즈라 미나야 스바하
			\item Om vajra minaya svaha
			\end{itemize}





\section{19수. 관세음보살 청련화수 진언}

			\begin{itemize}
			\item 다음 세상에 서방정토에서 태어나기를 원할 때
			\item 옴 기리기리 바아라 불반다 훔 바탁
			\item 범어 - 옴 끼(키)르끼(키)르 바(와)즈라 부르반다 훔파트
			\item Om kirkir vajra bhurvandha humphat
			\end{itemize}





\section{20수. 관세음보살 보경수 진언}

			\begin{itemize}
			\item 높고 큰 지혜를 얻고자 원할 때
			\item 옴 미보라 나락사 바라아 만다라 훔 바탁
			\item 범어 - 옴 비스푸라다 락사 바즈라 만달라 훔파트
			\item Om visphurada raksa vajra mandhala humphat
			\end{itemize}



 


\section{21수. 관세음보살 자련화수 진언}

			\begin{itemize}
			\item 부처님과 보살님을 친견하기를 원할 때
			\item 옴 사라사라 바아라 가라 훔 바탁
			\item 범어 - 옴 싸라싸라 바즈라 까라훔파트
			\item Om sarasara vajra karahumphat
			\end{itemize}





\section{22수. 관세음보살 보협수 진언}

			\begin{itemize}
			\item 땅속 깊이 묻혀 있는 온갖 보물을 얻고자 원할 때
			\item 옴 바아라 바사가리 아나맘나 훔
			\item 범어 - 옴 바즈라 빠사까리 가나맘라 훔
			\item Om vajra pasakari ganamamra hum
			\end{itemize}





\section{23수. 관세음보살 오색운수 진언}

			\begin{itemize}
			\item 한시바삐 불도를 성취하여 깨달음을 얻고자 원할 때
			\item 옴 바아라 가리라타 맘타
			\item 범어 - 옴 바즈라 까리라따 맘따
			\item Om vajra karirata mamta
			\end{itemize}





\section{24수. 관세음보살 군지수 진언}

			\begin{itemize}
			\item 다음 세상에는 천신이 되어 하늘에서 살고자 원할 때
			\item 옴 바아라 서가로타 맘타
			\item 범어 - 옴 바즈라 세카라루타 맘타
			\item Om vajra sekhararuta mamta
			\end{itemize}



\section{25수. 관세음보살 홍련화수 진언}

	 		\begin{itemize}
			\item 다음 세상에는 사람 몸 받지 않기를 원할 때
			\item 옴 상아례 사바하
			\item 범어 - 옴 샴그레 스바하
			\item Om samgre svaha
			\end{itemize}





\section{26수. 관세음보살 보극수 진언}

	 		\begin{itemize}
			\item 경쟁 상대나 원수의 힘을 없애고자 원할 때
			\item 옴 삼매야 기니하리 훔 바탁
			\item 범어 - 옴 삼마이야 끼니 하르 흠파트
			\item Om sammaiya kini har humphat
			\end{itemize}





\section{27수. 관세음보살수 보라수 진언}

	 		\begin{itemize}
			\item 언제 어디서나 호법신장들이 호위를 하기를 원할 때
			\item 옴 상아례 마하 삼만염 사바하
			\item 범어 - 옴 삼그레 마하 삼마얌 스바하
			\item Om samgre maha sammayam svaha
			\end{itemize}





\section{28수. 관세음보살 촉루장수 진언}

	 		\begin{itemize}
			\item 어떤 잡신들의 농간에도 휘둘리지 않고 뜻대로 지배하기를 원할 때
			\item 옴 도나 바아라 햑
			\item 범어 - 옴 두나 바즈라 하
			\item Om dhuna vajra hah
			\end{itemize}





\section{29수. 관세음보살 수주수 진언}

	 		\begin{itemize}
			\item 빨리 부처님께서 도와 주시기를 원할 때
			\item 나모라 다나다라 야야 옴 아나바제 미아예 싯디 싯달제 사바하
			\item 범어 - 나모 라뜨나뜨라야야 옴 아나바떼 비자야예 싣디싣다르테 스바하
			\item Namo ratnatrayaya om anabhate vijayaye sidhisiddharthe svaha
			\end{itemize}





\section{30수. 관세음보살 보탁수 진언}

	 		\begin{itemize}
			\item 아름답고 뛰어난 목소리 갖기를 원할 때
			\item 나모 바나맘 바나예 옴 아미리 담암베시리예 시리탐리니 사바하
			\item 범어 - 나모 빠드맘 빠나예 옴 암르땅감베 쉬르예 쉬르 땅르니 스바하
			\item namo padmam panaye om amrtamgambhe srye sr tamrni svaha
			\end{itemize}



 


\section{31수. 관세음보살 보인수 진언}

	 		\begin{itemize}
			\item 뛰어난 말솜씨와 글솜씨 갖기를 원할 때
			\item 옴 바아라녜 담아예 사바하
			\item 범어 - 옴 바즈라 네탐 자예 스바하(스와하)
			\item Om vajra netam jaye svaha
			\end{itemize}





\section{32수. 관세음보살 구시철구수 진언}

	 		\begin{itemize}
			\item 좋은 신들과 용왕이 보호하기를 원할 때
			\item 옴 아가로 다라가라 미사예 나모 사바하
			\item 범어 - 옴 아크로 따라까라 비사예 나모스바하
			\item Om akro tarakara visaye namosvaha
			\end{itemize}





\section{33수. 관세음보살 석장수 진언}

	 		\begin{itemize}
			\item 언제나 모든 생명체를 해치지 않기를 원할 때
			\item 옴 날지 날지 날타바지 날제 나야바니 훔 바탁
			\item 범어 - 옴 날티날티 날타파티 날테 다야빠니 훔파트
			\item Om nartinarti nartapati narte dayapani humphat
			\end{itemize}





\section{34수. 관세음보살 합장수 진언}

	 		\begin{itemize}
			\item 모든 존재들이 서로 존중하고 사랑하며 살기를 원할 때
			\item 옴 바나만 아링하리
			\item 범어 - 옴 빠드맘 그잘음 흐르
			\item Om padmam gjalm hr
			\end{itemize}





\section{35수. 관세음보살 화불수 진언}

	 		\begin{itemize}
			\item 태어날 때마다 부처님 곁을 떠나지 않기를 원할 때
			\item 옴 전나라 바맘타 이가리 나기리 나기리니 훔 바탁
			\item 범어 - 옴 짠(찬)다라 바맘따르 까르다끼르 다끼르니 훔파트
			\item Om candara bhamamtar kardakir dakirni humphat
			\end{itemize}





\section{36수. 관세음보살 화궁전수 진언}

	 		\begin{itemize}
			\item 태어날 때마다 늘 부처님 세계에서 태어나기를 원할 때
			\item 옴 미사라 미사라 훔 바탁
			\item 범어 - 옴 미사라 미사라 훔파트
			\item Om misara misara humphat
			\end{itemize}





\section{37수. 관세음보살 보경수 진언}

	 		\begin{itemize}
			\item 두루 널리 공부하여 잊지 않는 총명한 머리 갖기를 원할 때
			\item 옴 아하라 살바미냐 다라 바니제 사바하
			\item 범어 - 옴 아하라 사르바 비드야 다라 뿌디떼 스바하
			\item Om ahara sarva vidya dhara pudite svaha
			\end{itemize}





\section{38수. 관세음보살 불퇴금륜수 진언}

	 		\begin{itemize}
			\item 지금 이 몸으로 깨닫기 전까지 결코 물러서지 않기를 원할 때
			\item 옴 서나미자 사바하
			\item 범어 - 옴 사나미차(짜) 스바하
			\item Om sanamica svaha
			\end{itemize}





\section{39수. 관세음보살 정상화불수 진언}

	 		\begin{itemize}
			\item 나도 부처가 되리라는 흔들림 없는 확신을 갖고자 할 때
			\item 옴 바아라니 바아람예 사바하
			\item 범어 - 옴 바즈르니 바즈람게에 스바하
			\item Om vajrni vajramge svaha
			\end{itemize}





\section{40수. 관세음보살 포도수 진언}

	 		\begin{itemize}
			\item 풍요로운 과실수와 농산물 수확을 얻고자 할 때
			\item 옴 아마라 검제이니 사바하
			\item 범어 - 옴 아마라 깜떼디니 스바하
			\item Om amala kamtedini svaha
			\end{itemize}





\section{41수. 관세음보살 감로수 진언}

	 		\begin{itemize}
			\item 목마르고 배고픈 모든 중생이 겪는 고통을 없애고자 할 때
			\item 옴 소로소로 바라소로 바라소로 소로소로야 사바하
			\item 범어 - 옴 수루수루 보라수루 보라수루 수루수루예 스바하
			\item Om sulu sulu bholasulu bholasulu sulusuluye svaha
			\end{itemize}





\section{42수. 관세음보살 총섭천비수 진언}

	 		\begin{itemize}
			\item 어떠한 장애나 역경을 겪어도 반드시 모두 이겨내고자 할 때
			\item 다냐타 바로기제 새바라야 살바도따 오하야미 사마하
			\item 범어 - 따다탸 아바로끼데스바라야 싸르바 두시자 우아미예 스바하
			\item Tadyata avalokitesvaraya sarvadusiZa Uhamiye Svaha
			\end{itemize}



 

% ------------------------------------------------------------------------------
% End document
% ------------------------------------------------------------------------------
\end{document}




			\begin{table} [h]
	
			\caption{잔 골재의 표준입도}  
			\label{tab:title} 
	
			\begin{center}
			\tabulinesep=0.4em
			\begin{tabu} to 0.8\linewidth { X[r] X[c]  }
			\tabucline [1pt,] {-}
			체의 호칭 치수 (mm)		& 체를 통화한 것의 질량 백분율(\%) \\
			\tabucline [0.1pt,] {-}
			2.5	&100\\
			1.2	& 99 $\sim$ 100 \\
			0.6	& 60 $\sim$ 80 \\
			0.3	& 20 $\sim$ 50 \\
			0.15	&  5 $\sim$ 30 \\
			\tabucline [0.1pt,] {-}
			\end{tabu} 
			\end{center}
			\end{table}

